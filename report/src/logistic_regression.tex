\documentclass[b5paper,12pt]{jarticle}
\usepackage{amsmath,amssymb}
\usepackage{mathtools}
\usepackage{array}


\begin{document}
\section{ロジスティック回帰}

\subsection{分類問題}
ある入力(数値)からクラスに分類する問題。

\subsection{ロジスティック線形回帰モデル}
分類問題を解くための教師あり機械学習モデル
線形回帰モデルの出力をシグモイド関数に入力し、$y=1$となる確率を出力する。

\[
  P(Y = 1|\boldsymbol{x}) = \sigma(w_0 + \sum_{j_1}^{m}w_j x_j)
\]

\subsection{シグモイド関数}
入力は実数、出力は0~1の単調増加関数。
\[
  \sigma (x) = \frac{1}{1+\exp (-ax)}
\]

DNNでの活性化関数としてもよく用いられる。


\subsection{最尤推定}
ロジスティック回帰モデルでの最尤推定ではベルヌーイ分布を仮定する。
\[
  P(y) = p^y (1-p)^y
\]

\subsection{尤度関数}
データがある分布に従っていると仮定したとき、そのデータを生成したであろう分布の尤もらしさを表す関数。
\begin{align*}
  P(y_1, y_2, \cdots, y_n| w_0, w_1, \cdots, w_n) &= \prod_{i=1}^{n} p_i^{y_i} (1-p_i)^{1-y_i} \\
  &= \prod_{i=1}^{n} {\sigma (\boldsymbol{w}^T \boldsymbol{x}_i)}^{y_i} 
  (1-{\sigma (\boldsymbol{w}^T \boldsymbol{x}_i)})^{1-y_i} \\
  &= L(\boldsymbol{w})
\end{align*}
推定の際は総和で計算できるよう対数をとり、マイナスをかけた対数尤度関数$-\log L(\boldsymbol{w})$を最小化する。

\subsection{勾配降下法(SG)}
反復学習によりパラメータを逐次的に更新することで、対数尤度関数の最小化を行う。

パラメータが更新されなくなった場合、勾配が0になっているので、探索範囲内では最適な解が求められたことになる(ローカルミニマム)。
\subsection{確率的勾配降下法(SGD)}
データを一つずつランダムに選んでパラメータを更新する。

\subsection{分類の評価方法}
混同行列を使用して評価指標を計算する。

\subsubsection{再現率(Recall)}
検出に抜け漏れを少なくしたい場合に採用する。
\[
  \frac{TP}{TP+FN}
\]

\subsubsection{適合率(Precision)}
見逃しが多くても正確に検出したい場合に採用する。
\[
  \frac{TP}{TP+FP}
\]

\subsubsection{F値}
RecallとPrecisionの調和平均

\end{document}