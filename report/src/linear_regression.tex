\documentclass[b5paper,12pt]{jarticle}
\usepackage{amsmath,amssymb}
\usepackage{mathtools}
\usepackage{array}


\begin{document}
\section{線形回帰}

\subsection{回帰問題}
ある入力(離散、連続)から出力(連続)を予測する問題のこと。

(ex. 2次元の入出力データを一次関数で近似)

\subsection{線形回帰モデル}
回帰問題を解くための機械学習モデルの一つ。

予測タスクに対する教師あり学習でのアプローチで、
$m$次元の入力$x$からスカラー値の予測値$\hat{y}$を出力する。
\[
    \hat{y}=\boldsymbol{w^T x}+ w_0 = \sum_{j=1}^{m}w_j x_j + w_0
\]

\subsection{モデルのパラメータ}
モデルに含まれる推定すべき未知パラメータ。
線形回帰では最小二乗法で推定する。

\subsection{単回帰モデル}
$m = 1$の時の線形回帰モデル。
\[
  y = w_0 + w_1 x_1+ \epsilon
\]

\subsection{重回帰モデル}
$m > 1$の時の線形回帰モデル。
\[
  y = w_0 + w_1 x_1 + w_2 x_2+ \epsilon
\]

\subsection{モデルの検証}
モデルの汎化性能を測定するため、データを学習用データと検証用データに分離する。
(未知データに対してどれだけ精度よく推定できているかを測りたいため。)

\subsection{最小二乗法}
学習データの平均二条誤差(MSE)を最小とするパラメータを探索する。
MSEの勾配が0になるようなパラメータを求めればよい。


\end{document}