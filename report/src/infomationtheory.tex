\documentclass[b5paper,12pt]{jarticle}
\usepackage{amsmath,amssymb}
\usepackage{mathtools}
\usepackage{array}


\begin{document}
\section{情報理論}

\subsection{推定}
集団から一部を取り出し母集団の性質を推測すること。
\subsubsection{標本平均}
標本データの平均をとったもの。一致性と不偏性を持つため点推定によく用いられる。
\subsubsection{標本分散}
標本データの分散をとったもの。一致性は持つが不偏性は持たない。$\frac{n}{n-1}$をかけたものを不偏分散といい、これは不偏性を持つ。


\subsection{情報量}
あるできごと(事象)が起きた際、それがどれほど起こりにくいかを表す尺度である。
\subsubsection{自己情報量}
事象$x$が確率$P(x)$で起こるとき、自己情報量$I(x)$の定義を以下に示す。
\[
    I(x)=-\log(P(x))
\]
また、自己情報量の期待値をとったものをシャノンエントロピー$H(x)$といい、以下のように表される。
\[
    H(x)=E(I(x))=-\sum(P(x)\log P(x))
\]
\subsubsection{KLダイバージェンス}
2つの確率分布の違いを表すのに用いられる。確率分布$P(x)$、$Q(x)$のKLダイバージェンス$D_{KL}(P||Q)$の定義を以下に示す。
\[
    D_{KL}(P||Q)=E_{x\sim P}(-\log \frac{Q(x)}{P(x)})
\]
\[
    D_{KL}(P||Q)=\sum_x P(x) ((-\log Q(x))-(-\log P(x)))
\]

\subsubsection{交差エントロピー}
KLダイバージェンスの一部。確率分布$P(x)$、$Q(x)$の交差エントロピー$H(P,Q)$は以下のように表される。
\[
    H(P,Q)=E_{x\sim P}(-\log Q(x))=-\sum(P(x)\log Q(x))
\]


\end{document}